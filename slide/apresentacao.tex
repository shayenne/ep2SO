%%%%%%%%%%%%%%%%%%%%%%%%%%%%%%%%%%%%%%%%%
% Beamer Presentation
% LaTeX Template
% Version 1.0 (10/11/12)
%
% This template has been downloaded from:
% http://www.LaTeXTemplates.com
%
% License:
% CC BY-NC-SA 3.0 (http://creativecommons.org/licenses/by-nc-sa/3.0/)
%
%%%%%%%%%%%%%%%%%%%%%%%%%%%%%%%%%%%%%%%%%

%----------------------------------------------------------------------------------------
%	PACKAGES AND THEMES
%----------------------------------------------------------------------------------------

\documentclass{beamer}
\usepackage[utf8]{inputenc}
\usepackage[portuguese]{babel}
\usepackage{ragged2e}
\usepackage{url}
\mode<presentation> {

% The Beamer class comes with a number of default slide themes
% which change the colors and layouts of slides. Below this is a list
% of all the themes, uncomment each in turn to see what they look like.

%\usetheme{default}
%\usetheme{AnnArbor}
%\usetheme{Antibes}
%\usetheme{Bergen}
\usetheme{Berkeley}
%\usetheme{Berlin}
%\usetheme{Boadilla}
%\usetheme{CambridgeUS}
%\usetheme{Copenhagen}
%\usetheme{Darmstadt}
%\usetheme{Dresden}
%\usetheme{Frankfurt}
%\usetheme{Goettingen}
%\usetheme{Hannover}
%\usetheme{Ilmenau}
%\usetheme{JuanLesPins}
%\usetheme{Luebeck}
%\usetheme{Madrid}
%\usetheme{Malmoe}
%\usetheme{Marburg}
%\usetheme{Montpellier}
%\usetheme{PaloAlto}
%\usetheme{Pittsburgh}
%\usetheme{Rochester}
%\usetheme{Singapore}
%\usetheme{Szeged}
%\usetheme{Warsaw}

% As well as themes, the Beamer class has a number of color themes
% for any slide theme. Uncomment each of these in turn to see how it
% changes the colors of your current slide theme.

%\usecolortheme{albatross}
% Mesma cor do trabalho original
%\usecolortheme{beaver}
%\usecolortheme{beetle}
%\usecolortheme{crane}
%\usecolortheme{dolphin}
%\usecolortheme{dove}
%\usecolortheme{fly}
%\usecolortheme{lily}
%\usecolortheme{orchid}
%\usecolortheme{rose}
%\usecolortheme{seagull}
%\usecolortheme{seahorse}
\usecolortheme{whale}
%\usecolortheme{wolverine}

%\setbeamertemplate{footline} % To remove the footer line in all slides uncomment this line
\setbeamertemplate{footline}[page number] % To replace the footer line in all slides with a simple slide count uncomment this line

%\setbeamertemplate{navigation symbols}{} % To remove the navigation symbols from the bottom of all slides uncomment this line
}
\usepackage{caption}
\usepackage{graphicx} % Allows including images
\usepackage{booktabs} % Allows the use of \toprule, \midrule and \bottomrule in tables
%----------------------------------------------------------------------------------------
%	TITLE PAGE
%----------------------------------------------------------------------------------------

\title{EP2 - Gerenciador de Memoria} % The short title appears at the bottom of every slide, the full title is only on the title page

\author{Florence Alyssa \and Shayenne Moura} % Your name
\institute[USP] % Your institution as it will appear on the bottom of every slide, may be shorthand to save space
{
Sistemas Operacionais
 \\ Bacharelado em Ciência da Computação% Your institution for the title page
\medskip
\textit{} % Your email address
}
\date{19 de outubro de 2015} % Date, can be changed to a custom date





\begin{document}

\begin{frame}
\titlepage % Print the title page as the first slide
\end{frame}

\begin{frame}
\frametitle{Sumário}

\end{frame}


%\begin{frame}
%\frametitle{Visão Geral} % Table of contents slide, comment this block out to remove it
%\tableofcontents % Throughout your presentation, if you choose to use \section{} and \subsection{} commands, these will automatically be printed on this slide as an overview of your presentation
%\end{frame}

%----------------------------------------------------------------------------------------
%	PRESENTATION SLIDES
%----------------------------------------------------------------------------------------

%------------------------------------------------
\section{Objetivo} 
%------------------------------------------------

\begin{frame}
\frametitle{Objetivo}
Implementar um simulador de gerenciamento de memória com diversos algorítmos para gerência de espaço livre e substituição de página.

Linguagem: Python
\end{frame}

\begin{frame}
\frametitle{Suposições adotadas}
\begin{itemize}
\item processos ordenados segundo t0 de forma crescente
\item processos bem comportados (não escrevem onde não devem)
\item espaço suficiente na memória virtual
\item processos armazenados de forma integral e contínuas na memoria virtual
\item tempo de execução do processo: tf - t0
\end{itemize}

\end{frame}

\section{Shell}
\begin{frame}
\frametitle{Shell}

5 comandos: carrega, espaço, substitui, executa e sai.

> carrega [trace] 
  Cria uma lista para cada processo contendo os campos: 
  t0, nome, tf, espaço(byte), p1 t1, p2 t2, ... , pn tn

> espaco ou substitui
  Define variáveis de inicialização do gerenciador de espaço livre e algorítmo de substituição de páginas

> executa [intervalo]
  Cria listas de espaço livre e os arquivos ep2.mem e ep2.vir preenchidos com '-1'
  Simula e imprime do estado da memória a cada [intervalo]s
  
> sai
  Termina execução do simulador  

\end{frame}
%------------------------------------------------
%------------------------------------------------
\section{Visão Geral} 
%------------------------------------------------
\begin{frame}
\frametitle{Processos}

- Processo como objeto (com informação da lista)

- Thread por processo

- Sinal para inicialização

- Dormem ate t0

- Em t0: alocação de espaço na memrória virtual (semáforo)

- Preenchimento do espaço com pid

- Criação de dicionário: pid | base | limite
\justifying
\end{frame}


\begin{frame}
\frametitle{Processos}
IMAGEM
\justifying
\end{frame}


%------------------------------------------------
\section{Gerenciador de Espaço Livre} 
%------------------------------------------------
%------------------------------------------------

\subsection{First Fit}
%------------------------------------------------
\begin{frame}
\frametitle{First Fit}
- A busca sempre começa pela cabeça da lista
- Na lista de espaço livre verifica se lista[0] == 'L'
- Verifica se tem espaco em lista[2]
 

\justifying
\end{frame}

%------------------------------------------------
\subsection{Next Fit} 
%------------------------------------------------

\begin{frame}
\frametitle{Next Fit}

IMAGEM?
- A busca inicialmente começa pela cabeça da lista
- Nas buscas subsequentes começa de onde terminou
- Na lista de espaço livre verifica se lista[0] == 'L'
- Verifica se tem espaco em lista[2]


\justifying
\end{frame}

%------------------------------------------------
\subsection{Quick Fit}
%------------------------------------------------
\begin{frame}
\frametitle{Quick Fit}

\justifying
\end{frame}

\section{Substituiçao de Pagina}
%------------------------------------------------
\subsection{NRUP} 
%------------------------------------------------

\begin{frame}
\frametitle{Not Recently Used Page}

\justifying
\end{frame}




%------------------------------------------------
\subsection{FIFO} 
%------------------------------------------------

\begin{frame}
\frametitle{First-In First-Out}

\justifying

\end{frame}

%------------------------------------------------
\subsection{Second Chance} 
%------------------------------------------------

\begin{frame}
\frametitle{Second Chance}

\justifying
\end{frame}

%------------------------------------------------
\subsection{LRUP}
%------------------------------------------------
\begin{frame}
\frametitle{Least Recently Used Page}


\justifying
\end{frame}
%------------------------------------------------
\section{Resultados} 
%------------------------------------------------

\begin{frame}
\frametitle{NRU} 
\justifying
\end{frame}

\begin{frame}
\frametitle{FIFO} 
\justifying
\end{frame}

\begin{frame}
\frametitle{Second Chance} 
\justifying
\end{frame}

\begin{frame}
\frametitle{LRU} 
\justifying
\end{frame}


\end{document}
